%%% A template for a simple PDF/A file like a stand-alone abstract of the thesis.

\documentclass[12pt]{report}

\usepackage[a4paper, hmargin=1in, vmargin=1in]{geometry}
\usepackage[a-2u]{pdfx}
\usepackage[utf8]{inputenc}
\usepackage[T1]{fontenc}
\usepackage{lmodern}
\usepackage{textcomp}

\begin{document}

%% Do not forget to edit abstract in thesis.tex and thesis.xmpdata.

We can color knots by various finite quandles and check if they have non-trivial coloring. If so, we can
say that the knot is not an unknot. However, we will focus on quandles that always give trivial coloring.
It will turn out that they have interesting algebraic properties. In this work, we will show that a quandle
gives a trivial coloring for each knot if and only if the quandle is reductive, which is exactly when
the coloring invariant is Vassiliev's. We will make a similar characterization for links. That is,
a quandle gives a trivial coloring for each link if and only if it is a trivial quandle.

\end{document}
