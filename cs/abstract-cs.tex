%%% A template for a simple PDF/A file like a stand-alone abstract of the thesis.

\documentclass[12pt]{report}

\usepackage[a4paper, hmargin=1in, vmargin=1in]{geometry}
\usepackage[a-2u]{pdfx}
\usepackage[czech]{babel}
\usepackage[utf8]{inputenc}
\usepackage[T1]{fontenc}
\usepackage{lmodern}
\usepackage{textcomp}

\begin{document}

%% Do not forget to edit abstract in thesis.tex and thesis.xmpdata.

Uzly můžeme barvit různými konečnými quandly a zjišťovat, zda mají netriviální obarvení. Pokud ano, pak
dokážeme říct, že daný uzel není rozvázatelný. V práci se však zaměříme na takové quandly, které vždy dávají
triviální barvení. Ukáže se totiž, že mají totiž zajímavé algebraické vlastnosti. V této práci dokážeme,
že quandle dává pro každý uzel triviální barvení, právě tehdy když je quandle reduktivní, a to je právě tehdy,
když je barvicí invariant Vassilievův. Podobnou charakterizaci provedeme pro linky. Tedy quandle dává triviální
barvení pro každý link, právě tehdy když se jedná o triviální quandle.

\end{document}
