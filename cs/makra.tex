%%% Tento soubor obsahuje definice různých užitečných maker a prostředí %%%
%%% Další makra připisujte sem, ať nepřekáží v ostatních souborech.     %%%

%%% Drobné úpravy stylu

% Tato makra přesvědčují mírně ošklivým trikem LaTeX, aby hlavičky kapitol
% sázel příčetněji a nevynechával nad nimi spoustu místa. Směle ignorujte.
\makeatletter
\def\@makechapterhead#1{
  {\parindent \z@ \raggedright \normalfont
   \Huge\bfseries \thechapter. #1
   \par\nobreak
   \vskip 20\p@
}}
\def\@makeschapterhead#1{
  {\parindent \z@ \raggedright \normalfont
   \Huge\bfseries #1
   \par\nobreak
   \vskip 20\p@
}}
\def\fps@figure{h!}
\makeatother

% Citáty
\renewenvironment{quote}
               {\list{}{\rightmargin\leftmargin}%
                \item\relax\center\itshape\textquotedblleft\ignorespaces}
               {\unskip\unskip\textquotedblright\endlist}

% Toto makro definuje kapitolu, která není očíslovaná, ale je uvedena v obsahu.
\def\chapwithtoc#1{
\chapter*{#1}
\addcontentsline{toc}{chapter}{#1}
}

% Trochu volnější nastavení dělení slov, než je default.
\lefthyphenmin=2
\righthyphenmin=2

% Zapne černé "slimáky" na koncích řádků, které přetekly, abychom si
% jich lépe všimli.
\overfullrule=1mm
\let\OldRule\rule
\renewcommand{\rule}[2]{\OldRule{\linewidth}{#2}}

%%% Makra pro definice, věty, tvrzení, příklady, ... (vyžaduje baliček amsthm)

\theoremstyle{plain}
\newtheorem{veta}{Věta}
\newtheorem{lemma}[veta]{Lemma}
\newtheorem{tvrzeni}[veta]{Tvrzení}
\newtheorem{fakt}[veta]{Fakt}
\newtheorem{domnenka}[veta]{Domněnka}
\newtheorem{dusledek}[veta]{Důsledek}


\theoremstyle{definition}
\newtheorem{definice}{Definice}

\theoremstyle{remark}
\newtheorem*{pozn}{Poznámka}
\newtheorem*{priklad}{Příklad}
\newtheorem*{pozorovani}{Pozorování}

\newtheoremstyle{noparentheses}
  {\topsep}   % ABOVESPACE
  {\topsep}   % BELOWSPACE
  {\itshape}  % BODYFONT
  {0pt}       % INDENT (empty value is the same as 0pt)
  {\bfseries} % HEADFONT
  {.}         % HEADPUNCT
  {5pt plus 1pt minus 1pt} % HEADSPACE
  {\thmname{#1} \thmnumber{#2} \textnormal{\thmnote{#3}}}          % CUSTOM-HEAD-SPEC
\theoremstyle{noparentheses}
\newtheorem{vetac}[veta]{Věta}
\newtheorem{theoremc}[veta]{Věta}
\newtheorem{lemmac}[veta]{Lemma}
\newtheorem{tvrzenic}[veta]{Tvrzení}
\newtheorem{faktc}[veta]{Fakt}
\newtheorem{domnenkac}[veta]{Domněnka}
\newtheorem{dusledekc}[veta]{Důsledek}

%%% Prostředí pro důkazy

\newenvironment{dukaz}{
  \par\medskip\noindent
  \textit{Důkaz}.
}{
\newline
\rightline{$\qedsymbol$}
}

%%% Prostředí pro sazbu kódu, případně vstupu/výstupu počítačových
%%% programů. (Vyžaduje balíček fancyvrb -- fancy verbatim.)

\DefineVerbatimEnvironment{code}{Verbatim}{fontsize=\small, frame=single}

\providecommand{\tightlist}{%
  \setlength{\itemsep}{0pt}\setlength{\parskip}{0pt}}

\newcommand{\VerbBar}{|}
\newcommand{\VERB}{\Verb[commandchars=\\\{\}]}
\DefineVerbatimEnvironment{Highlighting}{Verbatim}{commandchars=\\\{\}}
% Add ',fontsize=\small' for more characters per line
\newenvironment{Shaded}{}{}
\newcommand{\AlertTok}[1]{\textcolor[rgb]{0.80,0.20,0.20}{\textbf{#1}}}
\newcommand{\AnnotationTok}[1]{\textcolor[rgb]{0.67,0.33,0.00}{\textbf{\textit{#1}}}}
\newcommand{\AttributeTok}[1]{\textcolor[rgb]{0.88,0.69,0.00}{#1}}
\newcommand{\BaseNTok}[1]{\textcolor[rgb]{0.80,0.00,0.00}{#1}}
\newcommand{\BuiltInTok}[1]{\textcolor[rgb]{0.99,0.00,0.69}{#1}}
\newcommand{\CharTok}[1]{\textcolor[rgb]{0.00,0.40,0.80}{#1}}
\newcommand{\CommentTok}[1]{\textcolor[rgb]{0.00,0.53,0.27}{\textit{#1}}}
\newcommand{\CommentVarTok}[1]{\textcolor[rgb]{0.00,0.53,0.27}{\textbf{\textit{#1}}}}
\newcommand{\ConstantTok}[1]{\textcolor[rgb]{0.33,0.33,0.33}{#1}}
\newcommand{\ControlFlowTok}[1]{\textcolor[rgb]{0.20,0.40,0.80}{\textbf{#1}}}
\newcommand{\DataTypeTok}[1]{\textcolor[rgb]{0.88,0.69,0.00}{#1}}
\newcommand{\DecValTok}[1]{\textcolor[rgb]{0.80,0.00,0.00}{#1}}
\newcommand{\DocumentationTok}[1]{\textcolor[rgb]{0.40,0.00,0.60}{\textit{#1}}}
\newcommand{\ErrorTok}[1]{\textcolor[rgb]{0.80,0.20,0.20}{\textbf{#1}}}
\newcommand{\ExtensionTok}[1]{#1}
\newcommand{\FloatTok}[1]{\textcolor[rgb]{0.80,0.00,0.00}{#1}}
\newcommand{\FunctionTok}[1]{\textcolor[rgb]{0.20,0.40,0.80}{#1}}
\newcommand{\ImportTok}[1]{\textcolor[rgb]{0.20,0.40,0.80}{#1}}
\newcommand{\InformationTok}[1]{\textcolor[rgb]{0.33,0.33,0.33}{\textbf{\textit{#1}}}}
\newcommand{\KeywordTok}[1]{\textcolor[rgb]{0.20,0.40,0.80}{\textbf{#1}}}
\newcommand{\NormalTok}[1]{#1}
\newcommand{\OperatorTok}[1]{\textcolor[rgb]{0.33,0.33,0.33}{#1}}
\newcommand{\OtherTok}[1]{\textcolor[rgb]{0.33,0.33,0.33}{#1}}
\newcommand{\PreprocessorTok}[1]{\textcolor[rgb]{0.00,0.53,0.27}{#1}}
\newcommand{\RegionMarkerTok}[1]{#1}
\newcommand{\SpecialCharTok}[1]{\textcolor[rgb]{0.00,0.53,0.27}{#1}}
\newcommand{\SpecialStringTok}[1]{\textcolor[rgb]{0.59,0.00,0.09}{#1}}
\newcommand{\StringTok}[1]{\textcolor[rgb]{0.59,0.00,0.09}{#1}}
\newcommand{\VariableTok}[1]{\textcolor[rgb]{0.00,0.53,0.27}{#1}}
\newcommand{\VerbatimStringTok}[1]{\textcolor[rgb]{0.00,0.53,0.27}{#1}}
\newcommand{\WarningTok}[1]{\textcolor[rgb]{0.80,0.20,0.20}{\textbf{\textit{#1}}}}

%%% Prostor reálných, resp. přirozených čísel
\newcommand{\R}{\mathbb{R}}
\newcommand{\N}{\mathbb{N}}
\newcommand{\Z}{\mathbb{Z}}
\newcommand{\Q}{\mathbb{Q}}
\newcommand{\C}{\mathbb{C}}
\newcommand{\F}{\mathbb{F}}
\newcommand{\K}{\mathcal{K}}
\newcommand{\Conj}[1]{\text{Conj}(#1)}
\newcommand{\Aut}[1]{\text{Aut}(#1)}
\newcommand{\Inn}[1]{\text{Inn}(#1)}
\newcommand{\Ima}{\text{Im}\,}
\newcommand{\Ker}{\text{Ker}\,}
\newcommand{\Col}[3][]{\text{Col}^{#1}_{#2}(#3)}
\newcommand{\Hom}[2]{\text{Hom}(#1,#2)}
\newcommand{\Orb}[2][]{\mathcal{O}^{#1}(#2)}
\newcommand{\Con}[1]{\text{Con}(#1)}

%%% Užitečné operátory pro statistiku a pravděpodobnost
\DeclareMathOperator{\pr}{\textsf{P}}
\DeclareMathOperator{\E}{\textsf{E}\,}
\DeclareMathOperator{\var}{\textrm{var}}
\DeclareMathOperator{\sd}{\textrm{sd}}

%%% Příkaz pro transpozici vektoru/matice
\newcommand{\T}[1]{#1^\top}

%%% Vychytávky pro matematiku
\newcommand{\goto}{\rightarrow}
\newcommand{\gotop}{\stackrel{P}{\longrightarrow}}
\newcommand{\maon}[1]{o(n^{#1})}
\newcommand{\abs}[1]{\left|{#1}\right|}
\newcommand{\dint}{\int_0^\tau\!\!\int_0^\tau}
\newcommand{\isqr}[1]{\frac{1}{\sqrt{#1}}}
\newcommand\quot[2]{
  \mathchoice
      {% \displaystyle
          \text{\raise1ex\hbox{$#1$}\Big/\lower1ex\hbox{$#2$}}%
      }
      {% \textstyle
          #1\,/\,#2
      }
      {% \scriptstyle
          #1\,/\,#2
      }
      {% \scriptscriptstyle  
          #1\,/\,#2
      }
}
%%% Vychytávky pro tabulky
\newcommand{\pulrad}[1]{\raisebox{1.5ex}[0pt]{#1}}
\newcommand{\mc}[1]{\multicolumn{1}{c}{#1}}
\newcommand{\bx}[1]{\tikz{\path[draw=#1,fill=#1] (0.125em,0.125em) rectangle (0.75em,0.75em);}}
